% SCIS 2024 原稿提出要領 (LaTeX用)

\documentclass{jarticle} % for platex
%\documentclass{ujarticle} % for uplatex
\usepackage{scis2024j}
\usepackage[hyphens]{url}


\begin{document}

\title{
  MLWEとMSIS仮定によるTwo-round n-out-of-n 署名プロトコルの実装\\
  Implementation of Two-round n-out-of-n Signing from Module-LWE and Module-SIS
}
\author{
  片岡 優蔵
  \samethanks{1}\\
  Yuzo Kataoka
  \thanks{
    京都大学大学院 情報学研究科 %
    〒606--8501 京都府京都市左京区吉田本町.  %
    Graduate School of Informatics Kyoto University,
    Yoshida-honcho, Sakyo-ku, Kyoto-shi, Kyoto, 606--8501, Japan. %
    kataoka.yuzo.24k@st.kyoto-u.ac.jp}\\
  \and
  原稿 書太郎
  \samethanks{1}\\
  Kaitaro Genko
}



\abstract*{ % 日本語あらまし
この文章では,SCIS 2024の原稿(WEB概要および論文)の作成とその提出方法について説明します.
SCIS 2024では,WEB概要(テキスト)を公開するとともに,論文を収録した概要集をSCIS 2024参加者のみアクセス可能なWEBからダウンロードを可能とする予定です.
著者の方々には\textbf{WEB概要}および\textbf{論文}を作成して頂きます.
WEB概要は500文字以内(日本語の場合)または1,300文字以内(英語の場合),論文は8ページ以内としてください.
WEB概要はテキスト入力による登録,論文は電子ファイルとして提出して頂くことになります.
\textbf{ハードコピーの提出は必要ありません.}
ご登録頂いたWEB概要はSCIS 2024のWEBページに,ご提出頂いた論文は概要集に,それぞれ収録されます.
WEB概要の公開および概要集発行の際のトラブルを防止するためにも,この文書に従って原稿を準備して頂き,ご提出頂きますようお願いいたします.
}
\keywords*{ %日本語キーワード
SCIS 2024,投稿原稿様式,ファイルフォーマット
}


\maketitle


\section{論文原稿(WEB概要)}
SCIS 2024では,SCIS 2024のWEBページに各論文のWEB概要を公開します.
WEB概要はWEB(発表申し込みページ)上でテキスト入力をお願い致します.
文字数は\textbf{500文字以内(日本語の場合)},または\textbf{1,300文字以内(英語の場合)}です.

参加申し込みを致しますと,マイページにアクセスすることができます.
マイページで講演の新規申し込みを行いますと,WEB概要をテキスト入力することができるようになります.



\section{論文原稿本文}

SCIS 2024では,ご提出頂いた論文については概要集に収録されます.
論文は\textbf{8ページ以内}での作成と提出をお願いいたします.

ご提出頂いた論文原稿についてはそのままの形式で収録されますので,下記の事項に注意して原稿を作成して下さい.
なお,この文書自身が概要集用の推奨原稿様式に従っています.


\subsection{サイズ・頁数の制限など}

用紙のサイズはA4とします.
原稿のページ数は8ページ以内です.

%アブストラクトPDFには,題目,著者,所属,住所,キーワード,あらましを記述してください.
%必要に応じて,図表や参考文献を記述していただいても構いません.

論文には,題目,著者,所属,住所,あらまし,キーワード,本文,図表,参考文献,附録などを記述してください.


\subsection{原稿のレイアウト}

上マージン17~[mm],下マージン20~[mm],左右マージン18~[mm]をとってください.
これらのマージン領域を除いた縦260~[mm],横174~[mm]の領域に収めて下さい.

また,原稿の1ページ目には,右上に縦16~[mm],横57~[mm]のSCISのラベルを記載してください.
ラベルの内容と様式は,この文書の上のSCISのラベルに従って下さい.
なお,論文の2ページ目以降にはラベルは必要ありません.
論文番号の記載は不要です.

論文にはページ番号を\textbf{入れてください.}


\subsection{原稿本文第1ページ目の構成}

本文の第1ページ目に必要な事項と言語について表~\ref{tab:const}にまとめました.

\begin{small}
\begin{table}
  \centering
    %\leavevmode
    \caption{論文第1ページ目の必要事項と言語}
    \begin{tabular}{|lcccc|} \hline
      本文の言語 & \multicolumn{2}{c}{日本語} & \multicolumn{2}{c|}{英語} \\
                 & 和文 & 英文 & 和文 & 英文 \\ \hline
      論文題目   & ○   & ○   & ×   & ○ \\
      著者名     & ○   & ○   & ×   & ○ \\
      所属       & ○   & ○   & ×   & ○ \\
      住所       & ○   & ○   & ×   & ○ \\
      あらまし   & △   & △   & ×   & ○ \\
      キーワード & △   & △   & ×   & ○ \\ \hline
    \end{tabular}
    \label{tab:const}

    ○:必須, △:和・英のどちらかを選択, ×:不要
\end{table}
\end{small}

概要集に収録する論文の「あらまし」は,WEB概要と同一内容である必要はありません(同一でも構いません).


\subsection{\LaTeX 用スタイルファイル}

原稿の作成に\LaTeX を使用される方は,\LaTeX スタイルファイルおよびテンプレートファイルを用意しましたので,\\
\texttt{https://www.iwsec.org/scis/2024/call.html} \\
より取得してください.
本文の言語により,英語版と日本語版があります.



\section{論文原稿の提出について}

\subsection{発表申込}

発表申し込みは,\\
\texttt{http://www.iwsec.org/scis/2024/registration.html}\\ の
「新規登録」を行い,マイページログイン後,「講演の新規申込」
で行ってください.


\subsection{原稿登録および投稿}

WEB概要および論文は,WEBでの電子登録および投稿による提出をお願い致します.
\textbf{ハードコピーの提出は必要ありません.}
WEBによる論文原稿の登録と投稿は,マイページログイン後に,画面の「論文原稿」でお願いします.

%アブストラクトPDFの提出期限は\textbf{2021年12月9日},
論文の提出期限は\textbf{2023年12月14日 17:00 JST}です.
論文原稿を登録および提出した後でも,締め切りまでは何度でも原稿を編集および差し替えが可能です.
WEB概要の公開および概要集の作成は,最後に登録されたWEB概要,投稿された論文ファイルを使用して行われます.
締め切り間際になりますと,サーバの混雑が予想されますので,\textbf{お早目の提出をお願いいたします}.
締め切り時点でシンポジウムのプログラム編成に必要な情報が提出されていない場合には,プログラムから除外されることがありますのでご注意ください.


\subsection{ファイルフォーマット}

論文のファイル形式は\textbf{PDF形式}に統一させていただきます.
ファイルサイズは1.5~MBを超過しないようにお願いいたします.
超過した場合,収録されない可能性があります.
特に,写真を多用される場合は留意してください.
また,下記の「利用可能なフォントについて」に留意してください.


\subsection{利用可能なフォントについて}

論文について,提出いただいた原稿の電子ファイルを「そのまま」収録します.
できるだけ多くの環境で収録原稿を読めるようにするために,利用フォントに関して下記のガイドラインに沿って頂くことを推奨します.

\begin{description}
  \item[欧文フォント]
    全てのフォントを埋めこんで頂くのが安全です.
  \item[和文フォント]
    ファイルサイズが超過しない限り,同様にPDFファイルに埋め込んでいただくことを推奨します.
    フォントを埋め込まない場合は,各種PDF閲覧・印刷ソフトで正しく表示・印刷できることを確認してください.
    特にWindows環境をお使いの方は,MacやUNIXなどの他環境でも正しい結果が得られるかどうか留意してください.
\end{description}

なお,フォント埋め込みの方法については,各PDF変換ソフトのマニュアルをご参照ください.


\section{問い合わせ先}

%\begin{tabular}{l}
%[
SCIS 2024 事務局

\url{scis2024-info-out@rdgml.intra.hitachi.co.jp}
%\end{tabular}

\begin{thebibliography}{9}
\bibitem{a}
著者名,\lq\lq 論文タイトル,'' 論文誌名,ページなど
\bibitem{b}
著者名,「本タイトル」, 出版社名,発行年など
\end{thebibliography}

\end{document}
% end of file
